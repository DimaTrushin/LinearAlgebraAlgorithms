\documentclass{article}

\usepackage[T1,T2A]{fontenc}
\usepackage[utf8]{inputenc}
\usepackage[english,russian]{babel}

\usepackage[left=2cm,right=2cm, top=2cm,bottom=2cm,bindingoffset=0cm]{geometry}

\usepackage{amsmath,amsthm,amssymb,amsfonts}
\usepackage{mathtext}
\usepackage{mathdots}
\usepackage[all]{xy}
\usepackage[colorlinks=true, urlcolor=blue]{hyperref}  

\newcommand{\tr}{\operatorname{\mathrm{tr}}}
\newcommand{\MatrixDim}[3]{\operatorname{\mathrm{M}_{#2\,#3}}(#1)}
\newcommand{\Matrix}[2]{\operatorname{\mathrm{M}_{#2}}(#1)}
\newcommand{\rk}{\operatorname{rk}}
\newcommand{\diag}{\operatorname{diag}}
\renewcommand{\Im}{\operatorname{Im}}

\renewcommand{\thesubsection}{\arabic{subsection}}

\begin{document}

\section*{Алгоритмы}

\subsection{Выделение базиса из системы векторов}

\paragraph{Дано}

Пусть $v_1,\ldots,v_m\in F^n$ -- вектора и $V = \langle v_1,\ldots,v_m\rangle$ -- их линейная оболочка.

\paragraph{Задача}

Среди векторов $v_1,\ldots,v_m$ найти базис пространства $V$ и разложить оставшиеся вектора по этому базису.

\paragraph{Алгоритм}

\begin{enumerate}
\item Запишем вектора $v_1,\ldots,v_m$ по столбцам в матрицу $A \in \MatrixDim{F}{n}{m}$.
Например, при $n = 3$, $m = 5$
\[
A = 
\begin{pmatrix}
{v_{11}}&{v_{21}}&{v_{31}}&{v_{41}}&{v_{51}}\\
{v_{12}}&{v_{22}}&{v_{32}}&{v_{42}}&{v_{52}}\\
{v_{13}}&{v_{23}}&{v_{33}}&{v_{43}}&{v_{53}}\\
\end{pmatrix}
\]

\item Приведем матрицу $A$ элементарными преобразованиями строк к улучшенному ступенчатому виду.
Например
\[
A' = 
\begin{pmatrix}
{1}&{0}&{a_{31}}&{0}&{a_{51}}\\
{0}&{1}&{a_{32}}&{0}&{a_{52}}\\
{0}&{0}&{0}&{1}&{a_{53}}\\
\end{pmatrix}
\]

\item Пусть $k_1,\ldots,k_r$ -- номера главных позиций в матрице $A'$.
Тогда вектора $v_{k_1},\ldots,v_{k_r}$ образуют базис $V$.
Например, в примере выше это вектора $v_1$, $v_2$ и $v_4$.

\item Пусть $v_i$ -- вектор соответствует неглавной позиции в $A'$.
Тогда в $i$-ом столбце $A'$ записаны координаты разложения $v_i$ через найденный базис выше.
Например, в примере выше $v_3 = a_{31}v_1 + a_{32}v_2$ и $v_5 = a_{51}v_1 + a_{52}v_2 + a_{53}v_4$.
\end{enumerate}

\paragraph{Пример}

Пусть 
\[
v_1 =
\begin{pmatrix}
{1}\\{3}\\{2}
\end{pmatrix},\,
v_2 =
\begin{pmatrix}
{1}\\{2}\\{1}
\end{pmatrix},\,
v_3 =
\begin{pmatrix}
{5}\\{12}\\{7}
\end{pmatrix},\,
v_4 =
\begin{pmatrix}
{1}\\{1}\\{1}
\end{pmatrix},\,
v_5 =
\begin{pmatrix}
{-1}\\{1}\\{0}
\end{pmatrix}\in F^3
\]
Тогда
\begin{gather*}
\begin{pmatrix}
{1}&{1}&{5}&{1}&{-1}\\
{3}&{2}&{12}&{1}&{1}\\
{2}&{1}&{7}&{1}&{0}\\
\end{pmatrix}\mapsto
\begin{pmatrix}
{1}&{1}&{5}&{1}&{-1}\\
{0}&{0}&{0}&{-1}&{2}\\
{2}&{1}&{7}&{1}&{0}\\
\end{pmatrix}\mapsto
\begin{pmatrix}
{1}&{1}&{5}&{1}&{-1}\\
{0}&{0}&{0}&{-1}&{2}\\
{1}&{0}&{2}&{0}&{1}\\
\end{pmatrix}\mapsto\\\mapsto
\begin{pmatrix}
{1}&{0}&{2}&{0}&{1}\\
{1}&{1}&{5}&{1}&{-1}\\
{0}&{0}&{0}&{1}&{-2}\\
\end{pmatrix}\mapsto
\begin{pmatrix}
{1}&{0}&{2}&{0}&{1}\\
{0}&{1}&{3}&{1}&{-2}\\
{0}&{0}&{0}&{1}&{-2}\\
\end{pmatrix}\mapsto
\begin{pmatrix}
{1}&{0}&{2}&{0}&{1}\\
{0}&{1}&{3}&{0}&{0}\\
{0}&{0}&{0}&{1}&{-2}\\
\end{pmatrix}
\end{gather*}
Тогда $v_1$, $v_2$ и $v_4$ -- базис линейной оболочки и $v_3 = 2v_1 + 3 v_2$ и $v_5 = v_1 - 2 v_4$.

\subsection{Нахождение какого-то базиса линейной оболочки}

\paragraph{Дано}

Пусть $v_1,\ldots,v_m\in F^n$ -- вектора и $V = \langle v_1,\ldots,v_m\rangle$ -- их линейная оболочка.

\paragraph{Задача}

Найти какой-нибудь базис подпространства $V$.

\paragraph{Алгоритм}

\begin{enumerate}
\item Уложить все вектора $v_i$ в строки матрицы $A\in \MatrixDim{F}{m}{n}$.

\item Элементарными преобразованиями строк привести матрицу к ступенчатому виду.

\item Ненулевые строки полученной матрицы будут искомым базисом.
\end{enumerate}

\subsection{Дополнение линейно независимой системы до базиса всего пространства стандартными векторами}

\paragraph{Дано}

Пусть $v_1,\ldots,v_m\in F^{n}$ -- линейно независимая система векторов, $V = \langle v_1,\ldots,v_m \rangle$ -- их линейная оболочка и $e_i$ -- стандартные базисные векторы, т.е. на $i$-ом месте стоит $1$, а в остальных $0$.

\paragraph{Задача}

Найти такие вектора $e_{k_1},\ldots, e_{k_{n-m}}$, что система $v_1,\ldots,v_m,e_{k_1},\ldots,e_{k_{n-m}}$ является базисом $F^{n}$.

\paragraph{Алгоритм}

\begin{enumerate}
\item Уложить вектора $v_i$ в строки матрицы $A\in\MatrixDim{F}{m}{n}$.

\item Привести матрицу $A$ к ступенчатому виду.

\item Пусть $k_1,\ldots,k_{n-m}$ -- номера неглавных столбцов.
Тогда $e_1,\ldots,e_{k_{n-m}}$ -- искомое множество.
\end{enumerate}

\subsection{Найти ФСР однородной СЛУ}

\paragraph{Дано}

Система однородных линейных уравнений $Ax = 0$, где $A\in \MatrixDim{F}{m}{n}$ и $x\in F^{n}$.

\paragraph{Задача}

Найти ФСР системы $Ax = 0$.

\paragraph{Алгоритм}

\begin{enumerate}
\item Привести матрицу $A$ элементарными преобразованиями строк к улучшенному ступенчатому виду.
Например
\[
A' = 
\begin{pmatrix}
{1}&{0}&{a_{31}}&{0}&{a_{51}}\\
{0}&{1}&{a_{32}}&{0}&{a_{52}}\\
{0}&{0}&{0}&{1}&{a_{53}}\\
\end{pmatrix}
\]

\item Пусть $k_1,\ldots,k_r$ -- позиции свободных переменных.
Если положить одну из этих переменных равной $1$, а все остальные нулями, то существует единственное решение, которое мы обозначим через $u_i$ (всего $r$ штук).
Например, для матрицы $A'$ выше свободные переменные имеют номера $3$ и $5$.
Тогда вектора (записанные в строку)
\[
u_1 = 
\begin{pmatrix}
{-a_{31}}&{-a_{32}}&{1}&{0}&{0}
\end{pmatrix},\,
u_2 = 
\begin{pmatrix}
{-a_{51}}&{-a_{52}}&{0}&{-a_{53}}&{1}
\end{pmatrix}
\]
являются ФСР.
\end{enumerate}

\subsection{Задать подпространство базисом, если оно задано матричным уравнением}

\paragraph{Дано}

Пусть $A\in\MatrixDim{F}{m}{n}$ и $V\subseteq F^{n}$ задано в виде $V = \{y\in F^{n}\mid A y = 0\}$.

\paragraph{Задача}

Найти базис подпространства $V$.

\paragraph{Алгоритм}

\begin{enumerate}
\item Найти ФСР системы $Ay = 0$.
Векторы ФСР будут базисом $V$.
\end{enumerate}

\subsection{Задать подпространство матричным уравнением, если оно задано линейной оболочной}

\paragraph{Дано}

Пусть $v_1,\ldots,v_k\in F^{n}$ -- набор векторов и $V = \langle v_1,\ldots,v_k \rangle$.

\paragraph{Задача}

Для некоторого $m$ найти матрицу $A\in\MatrixDim{F}{m}{n}$ такую, что $V = \{y\in F^{n}\mid Ay = 0\}$.

\paragraph{Алгоритм}

\begin{enumerate}
\item Уложить вектора $v_i$ в строки матрицы $B\in \MatrixDim{F}{k}{n}$.

\item Найти ФСР системы $Bz = 0$.

\item Уложить ФСР в строки матрицы $A\in \MatrixDim{F}{m}{n}$, где $m$ -- количество векторов в ФСР.
Матрица $A$ и будет искомой.
\end{enumerate}

\subsection{Найти матрицу замены координат}

\paragraph{Дано}

Векторное пространство $V$, $e=(e_1,\ldots,e_n)$ и $f = (f_1,\ldots,f_n)$ -- два базиса пространства $V$.
Известна матрица перехода от $e$ к $f$, т.е. $(f_1,\ldots,f_n) = (e_1,\ldots,e_n)A$, где $A\in \Matrix{F}{n}$.
Дан вектор $v = x_1e_1+\ldots+x_n e_n$.

\paragraph{Задача}

Найти разложение $v$ по  базису $f$.

\paragraph{Алгоритм}

\begin{enumerate}
\item Если $v = e x$, где $x\in F^{n}$, а также $v = f y$, где $y\in F^{n}$, то $y = A^{-1}x$.
\end{enumerate}

\subsection{Найти матрицу линейного отображения при замене базиса}

\paragraph{Дано}

Векторное пространство $V$ с базисами $e=(e_1,\ldots,e_n)$ и $e'=(e'_1,\ldots,e'_n)$, а также векторное пространство $U$ с базисами $f = (f_1,\ldots,f_m)$ и $f' = (f'_1,\ldots,f'_m)$.
Известны матрицы перехода $(e'_1,\ldots,e'_n) = (e_1,\ldots,e_n)C$ и $(f'_1,\ldots,f'_m) = (f_1,\ldots,f_m)D$, где $C\in \Matrix{F}{n}$ и $D\in \Matrix{F}{m}$.
Дано линейное отображение $\phi\colon V\to U$ заданное в базисах $e$ и $f$ матрицей $A\in\MatrixDim{F}{n}{m}$, т.е. $\phi e = f A$.

\paragraph{Задача}

Найти матрицу отображения $\phi$ в базисах $e'$ и $f'$, то есть такую $A'\in \MatrixDim{F}{n}{m}$, что $\phi e' = f' A'$.

\paragraph{Алгоритм}

\begin{enumerate}
\item $A' = D^{-1} A C$.
\end{enumerate}


\subsection{Определить существует ли линейное отображение заданное на векторах}

\paragraph{Дано}

Векторное пространство $V$ над полем $F$ и набор векторов $v_1,\ldots,v_k\in V$, векторное пространство $U$ и набор векторов $u_1,\ldots,u_k\in U$.

\paragraph{Задача}

Определить существует ли линейное отображение $\phi\colon V\to U$ такое, что $\phi(v_i) = u_i$.

\paragraph{Алгоритм}

\begin{enumerate}
\item Среди векторов $v_1,\ldots,v_k$ выделить линейно независимые, а остальные разложить по ним.

\item Пусть на предыдущем этапе базис получился $v_1,\ldots,v_r$, а $v_{r + i} = a_{i1} v_1 + \ldots + a_{ir}v_r$.

\item Искомое линейное отображение $\phi$ существует тогда и только тогда, когда выполняются равенства $u_{r+i} = a_{i1}u_1 + \ldots + a_{ir}u_r$.%
\footnote{В частности, если все $v_i$ оказались линейно независимыми, то линейное отображение $\phi$ обязательно существует.}
\end{enumerate}

\subsection{Найти базис образа и ядра линейного отображения}

\paragraph{Дано}

$\phi\colon F^{n}\to F^{m}$ задан $x\mapsto Ax$, где $A\in\MatrixDim{F}{m}{n}$.

\paragraph{Задача}

Найти базис $\Im \phi\in F^{m}$ и базис $\ker \phi\in F^{n}$.

\paragraph{Алгоритм}

\begin{enumerate}
\item Выделить базис среди столбцов матрицы $A$.
В результате получится базис $\Im \phi$.

\item Найти ФСР системы $Ax = 0$.
Полученная ФСР будет базисом $\ker \phi$.
\end{enumerate}

\subsection{Найти линейное отображение с заданными ядром и образом}

\paragraph{Дано}

Пространства $U\subseteq F^n$ и $W \subseteq F^m$ такие, что $\dim U + \dim W = n$.

\paragraph{Задача}

Найти матрицу линейного отображения $\varphi \colon F^n \to F^m$ такого, что $U = \ker \varphi$ и $W = \Im \varphi$.

\paragraph{Алгоритм}

\begin{enumerate}
\item Задать подпространство $W$ с помощью базиса.
Пусть $b_1,\ldots,b_k$ -- базис $W$.
Определим матрицу $B = (b_1 |\ldots |b_k)$.

\item Задать подпространство $U$ системой с линейно независимыми строками $U = \{y\in F^n \mid A y = 0\}$.

\item В силу условия $\dim U + \dim W = n$ матрица $A$ будет иметь столько же строк, сколько столбцов в матрице $B$.
В этом случае искомое линейное отображение задается матрицей $BA$.
\end{enumerate}

\subsection{Найти сумму подпространств заданных линейными оболочками}

\paragraph{Дано}

Подпространства $V,U\subseteq F^{n}$ заданные в виде $V = \langle v_1,\ldots,v_m\rangle$, $U = \langle u_1,\ldots,u_k\rangle$, где $v_i,u_j\in F^{n}$.

\paragraph{Задача}

Найти базис $V + U$.

\paragraph{Алгоритм}

\begin{enumerate}
\item Надо найти базис линейной оболочки $\langle v_1,\ldots,v_m,u_1,\ldots,u_k\rangle$.
\end{enumerate}

\subsection{Найти пересечение подпространств заданных линейными оболочками}

\paragraph{Дано}

Подпространства $V,U\subseteq F^{n}$ заданные в виде $V = \langle v_1,\ldots,v_m\rangle$, $U = \langle u_1,\ldots,u_k\rangle$, где $v_i,u_j\in F^{n}$.

\paragraph{Задача}

Найти базис $V\cap U$.%
\footnote{В это задаче можно задать подпространства системами, потом найти пересечение в виде системы, потом задать результат базисом.
Но есть куда более эффективный способ.}

\paragraph{Алгоритм}

\begin{enumerate}
\item Найти ФСР системы $D x = 0$, где $D = (v_1|\ldots|v_m|u_1|\ldots|u_k)$ и $x = \left(\frac{\alpha}{\beta}\right)$, где $\alpha\in F^{m}$, $\beta\in F^{k}$.

\item Пусть $\left(\left.\left.\frac{\alpha_1}{\beta_1}\right|\ldots\right|\frac{\alpha_s}{\beta_s}\right)$ -- ФСР.
Далее есть две опции (из них вторая опция предпочтительнее!):
\begin{itemize}
\item Множество векторов $R = (v_1|\ldots|v_m)(\alpha_1|\ldots|\alpha_s)$ порождает $V\cap U$.
Среди $(\alpha_1|\ldots|\alpha_s)$ можно выкинуть те $\alpha_i$, для которых $\beta_i = 0$.%
\footnote{Если ФСР построен по стандартному базису, то останутся $\alpha_i$ с нулевыми свободными переменными.}

\item Множество векторов $R' = (u_1|\ldots|u_k)(\beta_1|\ldots|\beta_s)$ порождает  $V\cap U$.
Причем можно рассматривать только ненулевые $\beta_i$.
\end{itemize}

\item Выделить базис среди столбцов $R$.
Это и будет базис $V\cap U$.
\begin{itemize}
\item Если векторы $u_1,\ldots,u_k$ были линейно независимы изначально и $\beta_i,\ldots,\beta_s$ -- все ненулевые сегменты ФСР с прошлого шага, то $(u_1|\ldots|u_k)(\beta_i|\ldots|\beta_s)$ будет базисом $V\cap U$.
\end{itemize}
\end{enumerate}

\subsection{Найти пересечение подпространств заданных матричным уравнением}

\paragraph{Дано}

Подпространства $V,U\subseteq F^{n}$ заданные в виде $V = \{y\in F^{n}\mid Ay = 0\}$, $U = \{y\in F^{n}\mid By = 0\}$, где $A\in \MatrixDim{F}{m}{n}$ и $B\in \MatrixDim{F}{k}{n}$.

\paragraph{Задача}

Задать $V\cap U$ в виде $\{y\in F^{n}\mid D y = 0\}$ для некоторого $D\in\MatrixDim{F}{k}{n}$, где $\rk D = k\leqslant n$.

\paragraph{Алгоритм}

\begin{enumerate}
\item Рассмотреть матрицу $D' = \left(\frac{A}{B}\right)$.

\item Выделить среди строк $D'$ линейно независимую подсистему.
Результат и будет искомая $D$.
\end{enumerate}

\end{document}